\documentclass[a4paper, 11pt]{article}

\usepackage{natbib}
\bibliographystyle{agsm}
\setcitestyle{authoryear,open={(},close={)}}

\input{layout.tex}

\title{\textbf{funding in the platform economy}\\Agency theory and venture capital finance}


\author{\textsc{Ans Vaessen}
\\13009722
\\{\textit{Entrepreneurship and Innovation}}}

\date{\today\\
word count: 3037}

%%%%%%%%%%%%%%%%% START %%%%%%%%%%%%%%%%%%

\begin{document}
\maketitle

\begin{abstract}
moet ik nog doen.
\end{abstract}



\vspace{30pt} % Some vertical space between the abstract and first section

\section*{Introduction}


With new businesses like Uber, AirBnB, GetYourGuide and Brewdog new busines models emerge from a closed business model to a more open business model within the sharing economy using online platforms to find customers. These four businesses not only use new busines models they also qualify as unicorns, ventures that are wort a \$billion or more \cite{TiddBessant}. Do these new business models impact the way start-ups are funded and what is the influence of this peer-to-peer sharing economy concept on the agency theory.

\section{small and medium entreprice businesses star-up finance}

New businesses or ventures are different from the more traditional business because they develop a new product which is not ready for the market, because it is still in a developing fase or the market is not aware of this new product. Therefore, \cite{TiddBessant} argue funding of a new business differs because it can not follow a normal cash flow from early sales. Also the initial funding needed depends on the sort of business,for example a biotechnology start-ups need more funding than a software start-up \citep{TiddBessant}. The sort of funding not only depends on the sort of business furthermore it depends on how far the business is developed. There are different models to explain the cycle a new business is going through. All the models follow more or less the same line. The following graph  \ref{fig:graph1} shows how a venture evolves from a seed, an initial idea, to the next stage and when the idea turns out to be succesful in the following stages it finally exits as an IPO, aquisition or merger.

\begin{figure}[h!]
    \centering
    \includegraphics[width=0.9\textwidth]{FinCycle.png}
    \caption{Start-up finance cycle.}
    \label{fig:graph1}
\end{figure}

Every stage also brings its own problems or challenges. The first stage the valley of death is hard to get through and most business fail in this so called Seed stage.
... meer uitleg misschien hoeveel... But if they survive they need more funding and that often comes from outside the company. \cite{casson2008oxford} states that the main source of funding for small firms are the more traditional institutes like banks, however young high tech businesses lack the track record or colletral requierd for such debt funding, so in order to grow the venture will look for outside equity funding.

This essay will focus on this form of finance in relation to the agency theory and how new business models, like open business, platform technology or sharing economy influence funding.

\section{equity funding}

Small and medium entrpreneurs will often need outside funding after the first initial fase, which is often funded with their own savings and money from family and friends. But once a business wants to grow further pass this inital start it needs more funding and the traditional investors like banks are not likely to invest because of the uncertainty and high riks of inheritent in new businesses \citep{Osnabrugge2000}. So instead of going to the bank an entrepreneur can choose to find an equity investor.

\subsection{Venture capital}

In the Oxford Handbook of Entrepreneurs the following definition is used:

\begin{quote}
"Venture capital investment consists in the purchase of shares of young, privatley held companies by outsiders for the primary purpose of capital gain" \citep[P.355]{casson2008oxford}.
\end{quote} 

The capital injections are mostly done in stages \cite{TiddBessant} for example distingues 4 stages because every stage of development needs different conditions and financial needs. As already mentioned in the initial stage money will come mostly from savings, family and friends but later on they need larger amounts of money. The idea is Venture capitalist invest in a high risk business after the initial stage but still early to sell the shares later for high profits when the company exits either to go on the public market, via aquisition or a merger \citep{TiddBessant}.

Venture Capital has a couple of advantages for the entrepreneur, it does not require tangible assets that banks ask for as a security, the investments are paid back when the company succeeds and is therefor more flexible and an other important advatage is the expertise the venture capatilist can bring into the company \citep{casson2008oxford}. However there is a downside as well. The venture capatalist wants part of the value of the company in return and also control in how the company is run \citep{casson2008oxford}.

Other ways to get more security for the venture capatalist is when a start-up has a patent for the new product or service.
\cite{nadeau2011innovation} shows there is evidence that protection of intelectual property rigths will improve a the changes of a succesful exit for the business and therefor it will be easier to attract venture capital.

 \cite{Roma} points that an other way to attract the interest of professional investors is, when entrepreneurs enjoy a large network that can stimulates trust, which can be accomplished through crowdfunding for example.


The venture capatalist wants control to make sure the outcome is what is aimed for, a profit when the company exits. The venture hardly ever provides a positive cashflow before a liquidity event (exit), the most common forms of exits are going public in a Initial Public Offerings (IPO) or a private sale of the venture to an other firm, Merger and Aquisition (M\&A) \citep{nadeau2011innovation}.

Venture capatalists can invest their own money but also raise funds from other investors (banks, pension funds or insurance companies) the so called fund providers \citep{casson2008oxford}.

Venture capital is well known but there are also two ways of equity funding like business angels and crowdfunding.

\subsection{Business Angels}
These are individuals that fund unlisted companies without having any family or friend connection, \citep{politis} adds that they not only invest capital, besides there money they also bring their business skills, personel networks and expertise. In grapgh \ref{fig:graph1} this is shown as a funding in the seed stage but they can still play an important part in later stages of the company.


\subsection{crowd funding}
Crowd funding is regarded as a relativly new way of to gain outside capital, mediated by a webportal a project or company can attratc mupltiple nonprofessionel investors \cite{TiddBessant}. \citep{belleflamme} distingueses two forms of crowdfunding either by pre order or advanced payment in exchange for a share in future profit. the author adds that crowdfunding  remains relativly small compared to other sources. Especially the pre-selling reshaped the funding lifecycle we saw earlier, \cite{bella} even states this could disrupt venture finance, because the company can sell millions of products without needing initial funds to produce them. 

\subsection{incubators and accelerators}
Programmes to help with your business an accelerators can help growth of an existing company focus on scale, whereas an incubators helps people with an idea build a company and focusses on innovation. An accelerator programme will make a small seed investment in return for equity in the entrepreneur receives funding but more importantly access to mentorship network to grow their business fast \citep{forrest}.

What ever source of funding is used their is always an asymetrie in information between the lender and company.

\section{Agency Theory}

In most companies that make use of outside funding there is a separation between
the owner of the business (agent) and the lender or investor this can be a
bank, venture capitalist or angel investor (principal). The principal wants
security in return of the investment to ensure the money invested is used well
and the principal is not harmed by actions of the agent \citep{jensen1976theory}.

The risk is in the asymmetry of information between the agent and the principal. Often the
agent has information about the firm and its day to day operations that is not
availble to the principle, and this can harm the principal if the agents uses the
information to benefit him or herself over the benefits of the principal
\citep{Osnabrugge2000}. 

\cite{Osnabrugge2000} continues to describes it is difficult for the principal to verify the information and that the two parties can have different views on how to run the company best, so they set up contracts to limit these so called agency costs.


These contracts are to tackle this problem, \citep{jensen1976theory} defines an agency relationship as contract in which both parties agree, the principal delegates some authority on decision making to the agent, to reduce the agency cost of informing each other on every detail, at the same time both parties agree on criteria to asses the preformance of the agent. 

Further reserach shows there is a difference in contracts. To form an optimal contract can be very costly so their is also the incomplete contract. The first optimal contaract is formulated to predict the forseeable future and follows behaviour vs outcome, screening and analysis are important to reduce the asymmetries of information between the principal and agent \citep{Osnabrugge2000}. The latter incomplete contract, presumes contract are always incomplete and it is better to invest in involvement and relationship with the agent \citep{Osnabrugge2000}. The author shows in his research that business angels prefer the incomplete contracts and ventrure capatalist choose to invest more in the screening in the pre-invesment fase in order to gain more security.

Furthermore, there is also an asymmetry in information between the fundprovider and the venture capatilist \citep{casson2008oxford}. ... nog een zinnetje....

Looking at crowdfunding principal-agent theory can be applied \citep{chaney} calls this inverted agency relationship, were consumers have some ideas about products to put on the market as principals and companies giving shape to these ideas as agents. So as with funding from venture capatalist, the company has to share its information and creative decision making with the consumer, \citep{chaney} unfortnuatly doesn't mention what this relationship implies for the next fase when a company looks for larger investment from for example venture capatilist.


\subsection{Staging and Syndication}
As mentioned earlier funding is done in stages and to grow a busines further it is possible that in a next round other venture capatilist are involved in invest in the same company this is called syndication of investment \citep{casson2008oxford}. Staging reduces agency cost as the venture capatalist can leave the venture if it is not performing well enough ans furthermore it is a way to get to know each other better before taking the next step \citep{colombo2016open}. There are different reasons for syndication  \cite{colombo2016open} points out three reasons for syndication, firstly to reduce risks, secondly to improve selection of high-quality ventures, the quality is checked by each syndicate member and thirdly to bring in differnet skills, expertise and networks in order to help the venture and add more value than with just one busines angel or venture capatalist.


\section{New business models}
In the past decades their is a trend towards more open innovation and open businesses. An other trend is the sharing economy some of the new businesses make extensive use of these new models and change the game for funding and information asymmetry.

\subsection{open business models}

Open busines models (OBS) are businesses that use external sources of technology and knowledge or sharing their own knowledge including intelctual properties with others outside \citep{chesbrough2007companies}. 
As \cite{colombo2016open} argues that the growing popularity of open innovation is not researched enough most models apply on closed business models. Open busines models (OBS) are businesses that use external sources of technology and knowledge or sharing their own knowledge including intelctual properties with others outside \citep{chesbrough2007companies}. As \cite{colombo2016open} points out there are many forms of open business models, one of them is Open Source Software. An other is the sharing economy making use of platform business models.


\subsection{sharing economy and platform business models}
A lot of companies build their open business model by using the online community, these business depend on the engagement of their community for value generation and expaning their business \citep{colombo2016open}. To share information the new ventures make use of platforms bringing together the producer and consumer. From simple retailing the market moved to online exchange of accommodation, information, transportation via platforms like Airbnb, GetYourGuide and Uber to name a view. There are many definitions but according to \cite{investopia} this economic model is frequently defined as peer-to-peer (P2P) based activity of acquiring, providing or sharing access to goods and services that are facilitated by a community based on-line platform.



\section{Role of Venture Capital for platform business models}

\cite{colombo2016open} explains that Venture capatalist are best suited to deal with open business models, they are high risk and venture capatalist can cope well with information asymetry. In closed business models for software, entrepreneurs often rely on Patents to protect their product and prevent others copy it and to capture the value by licening the product for example software code, when entrepreneurs adopt an open business model the value comes from selling beter versions or extra services on top of the free software \citep{colombo2016open}. In his reseach the focus is on open source software companies not the platform models.   


There are several stories about these new business models and there way to the public market. To get a better idea and to see if there are any pattern comparable with \cite{colombo2016open} research towards staging four cases of platform business models will be discussed briefly.


\subsection{Uber} 

The company is a platform which connects drivers with people who are looking for a ride. \cite{griswold} writes how Uber was founded when a few friends came up with the idea in 2009 and launched the app with in 2010 after some issues they were forced to drop the word Cab because not a taxi service and continued as UBER, they rolled out to many cities and countries.

The competition grew and they spend a lot of money buying out the competition in China and India. In return Uber’s regional rivals in Asia had powerful investors as well like Softbank and Alibaba. They won the American market but nog the Asian one and sold their China operations. But even though they made a small profit after that in 2017, Uber is continuously unprofitable. In 2017, Uber lost over 4 billion dollar in 2017 and only was profitable in 2018 because it sold international operations. The admit they still rely on discounts for consumers and driver incentives and is planning to continue these subsidies for customers when they went public in April. With going public they hope to convince prospective investors that they can be the next big company \cite{griswold}.

\subsection{AirBnB} 
The company developed a platform website connecting people who look for accommodation with people who want to rent-out a accommadation. A good example of an open business model were the input comes directly from the providers and sharing this information on a platform. The company was founded in 2008 and made ust of an incubeter programm, Y-Combinator, they invested \$20.000 seed money before raising money from various angel investors, and Venture capatalists, vallued in 2018 at \$31 billion \citep{mazzarini}.

\subsection{GetYourGuide} 
This platform is a booking platform for tours, attractions and activities founded by a group of students who came up with the idea after a trip to China \citep{getyourguide}. The first years the company survive on bootstrapping its first seed round in 2012 raised \$2 million. This was followed by a Series A in early 2013 when it raised \$14 million \cite{webintravel}. Now they are in a late stage venture funding series E \citep{crunch}.

\subsection{Brewdog} 
This is a brewery and the company might not look like a platform but are called a new-age business model, starting as a crowdfunded business in 2007 by sellting shares as Equity Punk, even with new investment rounds they kept the crowdfunding model, Equity Punk partners could vote in advance on the proposed TSG investment \citep{danziger}.

\section{is there a pattern}

Out of four cases a couple of observations can be made as we can see in the table  \ref{fig:graph2} using the data from \citep{crunch} to make a comparison between the 4 comanies.


\begin{figure}[h!]
    \centering
    \includegraphics[width=0.9\textwidth]{table.png}
    \caption{compare.}
    \label{fig:graph2}
\end{figure}




It is only a brief look at four companies and in no way represents any out come but comparing with and more reseracht necesary
tabel


\section{agency theory and information asymetry in platforms}
As explained earlier the agency theory in regards to the investor and the company explains how there is a asymetry in information. There is plenty of information about the asymetry of information between seller and buyer on the online market and how peer-to-peer business closes this gaps with reviews about both the customers and the producer or service provider. Hower ever I have not found any resarch about the inmapct of this more open new business model on the information asymystry towards the investors. Only  \cite{colombo2016open} briefly touches on subject by asusming that when more are involved one can assume the information is even more asymetrical. 




Different compnies different stages hard to compare but intersting to see the funding is 




\section{Conclusion}

And to conclude one could say bla bla bla


%% disable some things
\renewcommand{\textbf}{}
\renewcommand{\bf}{}
\bibliography{biblio}{}
\end{document}
